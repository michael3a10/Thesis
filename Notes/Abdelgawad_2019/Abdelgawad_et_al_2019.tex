\chapter{Optimization of device geometry in single-plate DMF}
\section{Abstract Summary}
\begin{enumerate}
    \item Combination of numerical simulations and experimental tests to compare six different single-plate (open-type) designs.
    \item \textbf{\emph{COMSOL}}, via finite element analysis, is used to:
        \begin{enumerate}
            \item \textbf{\emph{Calculate electrodynamic actuation forces in each designs}}
        \end{enumerate}
    \item \textbf{Forces predicted by the electrodynamic model} were \textbf{\emph{in agreement}} with \textbf{forces predicted using electromechanical models}
\end{enumerate}

\section{Problems to be solved}
\begin{enumerate}
    \item There is no predefined geometry for positioning ground electrodes relative to actuation electrodes in single plate devices.
    \item Changing the position of the grounding electrodes/wires on the device will change the intensity and distribution of the electric field in the vicinity of the droplet should:
    \begin{enumerate}
        \item Change the actuation force on the droplet
        \item Maximum droplet speed that can be achieved
    \end{enumerate}
    \item However, until now, there have been no studies on the optimum design of such devices possibly due to:
    \begin{enumerate}
        \item Time and cost that would be required to experimentally evaluate the different geometries
        \item Lack of accessible and verifiable numerical modeling tools that can compare the different designs
        \item Use in-house written codes 
        \item Available source codes that require the writing of long and complex scripts
    \end{enumerate}
\end{enumerate}

\section{Proposed Method}
\subsection{Preparation}
\begin{enumerate}
    \item To model in COMSOL, use \textbf{\emph{electrodynamic}} instead of electrowetting or electromechanical models
    \begin{itemize}
        \item Calculate actuation forces on droplets approximated as spherical caps with unchanging geometry (no wetting or contact angle change)
    \end{itemize}
    \item Electrodynamic and electrowetting/mechanical models are not mutually \newline exclusive, as contact angle change in electrowetting can be interpreted as an equilibrium between electrodynamic forces and surface tension.
    \item Solve for the electric field around the droplet and calculated the charge \newline density accumulated on droplet surfaces for six single-plate designs \newline reported in the literature.
    \item Use Maxwell-stress tensor formulation to calculate the droplet actuation forces in each design.
    \item Validate the modeling results experimentally by estimating the forces acting on droplets in devices corresponding to three of the modeled designs using a unique measurement technique.
    \item Modeling results were used to generate a list of
    design tips for production of devices with maximum actuation forces.
\end{enumerate}
\subsection{Geometrical Parameters}
\section{Results and Findings}
\section{Discussion and Critique}
\section{Applications to My work}