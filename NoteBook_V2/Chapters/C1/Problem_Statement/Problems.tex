\section{Problem Statement}
The persistence of salmonella contamination poses a threat to public health, leading to a significant number of foodborne illnesses worldwide. Identifying which salmonella species contributed to an outbreak are often impeded by the inherent limitations of the conventional methods employed for detecting this pathogen such as time-consuming and labor-intensive procedures \cite{silvaRecentDevelopmentsLateral2023}.\\

These techniques also require specialized laboratory equipments, trained staff and large volumes of samples, which limits their usefulness in a variety of contexts, particularly those with low resources or in times of emergency \cite{shenBiosensorsRapidDetection2021,wangOverviewRapidDetection2021}.\\

To address these issues, microfluidic techniques have been employed to reduce the volume of samples and reagents needed for bacteria sample preparation, fast reaction and automatic operation \cite{qiMicrofluidicBiosensorRapid2021,nguyenCompleteProtocolRapid2019} prior for detection procedures via electrochemical (EC) or optical sensors. \\

Microfluidics techniques however involves in designing complex geometries, inclusion of micropumps and microvalves to prepare the samples and produce the necessary results which can be time-consuming \cite{suMicrofluidicsBasedBiochipsTechnology2006}. Therefore, it's necessary to find a different approach when creating a platform that can prepare samples and identify \emph{Salmonella} species without relying on  extra hardware or intricate geometries.\\

DMF is a liquid handling technology which manipulates fluids into discrete droplets on a surface of an array of electrodes through non-contact forces such as electrical, magnetic or thermal \cite{nguyenCompleteProtocolRapid2019,qiMicrofluidicBiosensorRapid2021}.\\

Compared to the microfluidics approach, this method would enable sample volumes to be lowered to nanoliters. Additionally, because the generated sample droplets can be precisely manipulated, samples in droplet form can be further divided into smaller droplets for parallel detection.\\

For bacteria samples identification, optical sensors are often used as compared to EC sensors when combined with DMF-based platform due to the EC sensor lifetime contributed by the rapid dissolution of reference electrode (RE) made up of silver chloride (\ce{AgCl}) \cite{farzbodIntegrationReconfigurablePotentiometric2018}.\newpage

These systems are, however, expensive and not portable for conducting sample preparation and identification for on-site detection \cite{andersonThinfilmtransistorDigitalMicrofluidics2021}. Hence, this study would investigate and develop a DMF-based platform to be integrated with an array of improved EC sensors to perform \emph{Salmonella} bacteria species detection.\\

Previous works are concentrated on detecting a specific pathogen from a droplet \cite{andersonThinfilmtransistorDigitalMicrofluidics2021,foudehRapidMultiplexDetection2015,luSensitiveAutomatedDetection2023 ,sistaDigitalMicrofluidicPlatform2011} whereby a singular peak detected during the identification process or changes in colour signifies the pathogen existence in the sample. There are ,however ,  less reports on detecting multiple species of a targeted pathogen such as \emph{Salmonella} bacteria.\\

To address this, a probability method such as Fuzzy Logic can be introduced to help in detecting the probability of \emph{Salmonella} bacteria species extisted within the prepared sample. The capability of Fuzzy Logic in determining the pathogen species should be investigated in terms of the sample concentration and voltage resulted from the EC sensor read-out.
