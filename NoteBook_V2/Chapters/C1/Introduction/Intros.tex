\chapter{Introduction}
\section{General Overview}
Salmonella contamination poses a persistent threat to public health, leading to a significant
number of foodborne illnesses worldwide. The conventional methods employed for Salmonella detection often suffer from inherent limitations, hindering the timely and accurate identification of this pathogen. Current techniques, such as culture-based methods and
polymerase chain reaction (PCR), are resource-intensive, time-consuming, and may lack
the sensitivity required for early detection.Additionally, these methods often demand specialized
laboratory equipment and skilled personnel, limiting their applicability in diverse settings, especially in resource-limited environments or during urgent situations.\\

The need for a rapid, cost-effective, and user-friendly Salmonella bacteria detection system is evident, given the severe health consequences associated with delayed identification. Conventional approaches may result in prolonged response times, contributing to the spread of infections and complicating public health interventions. Furthermore, the increasing global interconnectedness of food supply chains necessitates the development of technologies that
can be easily deployed in various settings, including laboratories, agricultural facilities, and field environments.\\

Addressing these challenges requires a paradigm shift towards innovative detection methodologies
that not only enhance sensitivity and specificity but also improve accessibility and affordability. The development of a scalable and portable Digital Microfluidics Platform (DMF) integrated with electrochemical (EC) sensors could provide an insight of which Salmonella bacteria species causes an outbreak.\\

This project aims to contribute to the advancement of public health initiatives and ensure the timely identification of Salmonella, thereby mitigating the impact of foodborne outbreaks
on a global scale.

\section{Problem Statement}
\section{Research Objectives}
\section{Notes Layout}