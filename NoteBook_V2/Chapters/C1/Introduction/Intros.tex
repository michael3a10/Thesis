\chapter{Introduction}
\section{General Overview}
The widespread occurrence of Salmonella contamination represents an ongoing public health challenge, causing numerous foodborne illnesses globally. Traditional Salmonella detection methods have significant limitations that prevent quick and precise pathogen identification.\\

Established techniques like culture-based approaches and PCR are resource-demanding, slow, and sometimes lack the necessary sensitivity for early detection. These methods typically require specialized equipment and trained technicians, restricting their use across different environments, particularly in settings with limited resources or during emergency situations.\\

There is a clear demand for Salmonella detection systems that are quick, economical, and simple to use, considering the serious health implications of delayed identification. Conventional methods often result in extended response times, potentially facilitating infection spread and hampering effective public health responses.\\

Additionally, as food supply chains become increasingly interconnected worldwide, there is a growing need for detection technologies that can be readily implemented in various contexts, including laboratories, agricultural settings, and field locations.\\

To overcome these obstacles, a fundamental shift toward innovative detection approaches is necessary with respect to methods that allows in-situ real-time verification,enhance sensitivity and specificity while also improving accessibility and cost-effectiveness.\\

Developing a versatile and compact Digital Microfluidics Platform (DMF) with integrated electrochemical (EC) sensors could help identify specific Salmonella bacterial species responsible for outbreaks.\\

This initiative seeks to enhance public health strategies and enable prompt Salmonella identification, thereby reducing the global impact of foodborne outbreaks.\\


\begin{figure}[htbp]
    \centering
    \begin{tikzpicture}[
        mindmap,
        concept color=red!30,
        level 1 concept/.append style={font=\large\bfseries, level distance=4.5cm, sibling angle=90},
        level 2 concept/.append style={font=\normalsize, level distance=4cm, sibling angle=40},
        every node/.style={concept, execute at begin node=\hskip0pt, text width=2.8cm, align=center, font=\small},
        grow cyclic,
        nodes={concept},
    ]
        
        \node[font=\large\bfseries, text width=3cm] {Salmonella Detection}
        [clockwise from=45]
        child[concept color=orange!50] { node[font=\bfseries] {Traditional Methods}
            [clockwise from=110]
            child { node {Culture-based approaches} }
            child { node {PCR methods} }
            child { node {Specialized equipment needed} }
            child { node {Slow processing time} }
        }
        child[concept color=blue!50] { node[font=\bfseries] {Challenges}
            [clockwise from=20]
            child { node {Limited sensitivity} }
            child { node {Resource intensive} }
            child { node {Requires trained technicians} }
            child { node {Delayed response time} }
        }
        child[concept color=green!50] { node[font=\bfseries] {Needs}
            [clockwise from=290]
            child { node {Quick detection} }
            child { node {Cost-effective solutions} }
            child { node {Field-deployable technologies} }
            child { node {Global implementation} }
        }
        child[concept color=purple!50] { node[font=\bfseries] {Proposed Solution}
            [clockwise from=200]
            child { node {Digital Microfluidics Platform} }
            child { node {Electrochemical sensors} }
            child { node {In-situ real-time verification} }
            child { node {Enhanced sensitivity \& specificity} }
        };
    \end{tikzpicture}
    \caption{Mind map illustrating the challenges of traditional Salmonella detection methods and the proposed Digital Microfluidics Platform solution with integrated electrochemical sensors.}
    \label{fig:salmonella-detection}
\end{figure}
